
% LaTeX Vorlage für wissenschaftliche Arbeiten am IGMR 
% LaTeX template for thesis at the IGMR
% -------------------------------------------------------------------------
%
%		AUTHOR: 		Schoeler, Frederic (FS)
%		LAST CHANGE:	2017-11-16
%		VERSION:		2.1
%
% -------------------------------------------------------------------------
% 		ÄNDERUNGSVERZEICHNIS / List of changes
%
% 		V 1.0 | 2015-02-04 | F.Schoeler     | First english version
%		V 1.1 | 2016-01-07 | F.Schoeler     | One template for german and english
%		V 1.2 | 2017-03-21 | F.Schoeler     | Small changes for compatibility with TexLive 2016
%		V 2.0 | 2017-05-08 | F.Schoeler     | Introduction of igm.sty
%		V 2.1 | 2017-11-16 | F.Freikwoski   | Migration to IGMR
%
% -------------------------------------------------------------------------
%		AKTUELLE Probleme / CURRENT problems: 
%
% -------------------------------------------------------------------------
% 		LITERATUREMPFEHLUNG: / RECOMMENDED LITERATURE: 
%
% 		Kopka, Helmut: 	LaTeX - Band 1: Einführung, 
% 			Published by Addison-Wesley, Third Edition,  2002
%			Available at the textbook collection: ST 351T28 0001-1+3
%		Schlosser, Joachim:	Wissenschaftliche Arbeiten schreiben mit LATEX
%			Published by mitp, Third edition, 2009
%			Available at the textbook collection: ST 351T28 0009+3
%
%		LATEX-prohibitions:		root/Latex/Literatur/l2tabu.pdf
%		Avoid Eqnarray!:		root/Latex/Literatur/avoideqnarray.pdf	
%		Manual KomaScript:		root/Latex/Literatur/scrguide.pdf
%
% -------------------------------------------------------------------------
% 		HINWEISE / PLEASE NOTE: 
%
%		In order to update the table of contents it is necessary to compile twice.
%		After the first process of compiling, Latex saves the data to a document 
%		on the hard drive and imports the data only upon a second process of compiling. 	
%		Every update to the table of contents involves compiling pdflatex, biblatex and again pdflatex. 	
%
% -------------------------------------------------------------------------
%		Magic Comments
%		
% !TeX TXS-program:bibliography = txs:///biber 
%
% -------------------------------------------------------------------------
%
\documentclass[
	english,			% ngerman / english	, 
	draft 	= false,	% [final/draft]			Document status
	twoside	= false,	    % [false/true]		single-sided document
	fleqn				% {equation} left justify
	]{scrbook}           % Koma-Script

% --------------------------------------------------------------------------
% 		Pakete / Packages 
% --------------------------------------------------------------------------
\usepackage{igm}

\definecolor{rwth_blau100}{RGB}{0,84,159}
\definecolor{rwth_blau75}{RGB}{64,127,183}
\definecolor{rwth_blau50}{RGB}{142,186,229}
\definecolor{rwth_blau25}{RGB}{199,221,242}
\definecolor{rwth_blau10}{RGB}{232,241,250}

\definecolor{rwth_schwarz100}{RGB}{0,0,0}
\definecolor{rwth_schwarz75}{RGB}{100,101,103}
\definecolor{rwth_schwarz50}{RGB}{156,158,159}
\definecolor{rwth_schwarz25}{RGB}{207,209,210}
\definecolor{rwth_schwarz10}{RGB}{236,237,237}

\definecolor{rwth_magenta100}{RGB}{227,0,102}
\definecolor{rwth_magenta75}{RGB}{233,96,136}
\definecolor{rwth_magenta50}{RGB}{241,158,177}
\definecolor{rwth_magenta25}{RGB}{249,210,218}
\definecolor{rwth_magenta10}{RGB}{253,238,240}


\definecolor{rwth_gelb100}{RGB}{255,237,0}
\definecolor{rwth_gelb75}{RGB}{255,240,85}
\definecolor{rwth_gelb50}{RGB}{255,245,155}
\definecolor{rwth_gelb25}{RGB}{255,250,209}
\definecolor{rwth_gelb10}{RGB}{255,253,238}

\definecolor{rwth_petrol100}{RGB}{0,97,101}
\definecolor{rwth_petrol75}{RGB}{45,127,131}
\definecolor{rwth_petrol50}{RGB}{125,164,167}
\definecolor{rwth_petrol25}{RGB}{191,208,209}
\definecolor{rwth_petrol10}{RGB}{230,236,236}

\definecolor{rwth_türkis100}{RGB}{0,152,161}
\definecolor{rwth_türkis75}{RGB}{0,177,183}
\definecolor{rwth_türkis50}{RGB}{137,204,207}
\definecolor{rwth_türkis25}{RGB}{202,231,231}
\definecolor{rwth_türkis10}{RGB}{235,246,246}

\definecolor{rwth_grün100}{RGB}{87,171,39}
\definecolor{rwth_grün75}{RGB}{141,192,96}
\definecolor{rwth_grün50}{RGB}{184,214,152}
\definecolor{rwth_grün25}{RGB}{221,235,206}
\definecolor{rwth_grün10}{RGB}{242,247,236}

\definecolor{rwth_maigrün100}{RGB}{189,205,0}
\definecolor{rwth_maigrün75}{RGB}{208,217,92}
\definecolor{rwth_maigrün50}{RGB}{224,230,154}
\definecolor{rwth_maigrün25}{RGB}{240,243,208}
\definecolor{rwth_maigrün10}{RGB}{249,250,237}

\definecolor{rwth_orange100}{RGB}{246,168,0}
\definecolor{rwth_orange75}{RGB}{250,190,80}
\definecolor{rwth_orange50}{RGB}{253,212,143}
\definecolor{rwth_orange25}{RGB}{254,234,201}
\definecolor{rwth_orange10}{RGB}{255,247,234}

\definecolor{rwth_rot100}{RGB}{204,7,30}
\definecolor{rwth_rot75}{RGB}{216,92,65}
\definecolor{rwth_rot50}{RGB}{230,150,121}
\definecolor{rwth_rot25}{RGB}{243,205,187}
\definecolor{rwth_rot10}{RGB}{250,235,227}

\definecolor{rwth_bordeaux100}{RGB}{161,16,53}
\definecolor{rwth_bordeaux75}{RGB}{182,82,86}
\definecolor{rwth_bordeaux50}{RGB}{205,139,135}
\definecolor{rwth_bordeaux25}{RGB}{229,197,192}
\definecolor{rwth_bordeaux10}{RGB}{245,232,229}

\definecolor{rwth_violett100}{RGB}{97,33,88}
\definecolor{rwth_violett75}{RGB}{131,78,117}
\definecolor{rwth_violett50}{RGB}{168,133,158}
\definecolor{rwth_violett25}{RGB}{210,192,205}
\definecolor{rwth_violett10}{RGB}{237,229,234}

\definecolor{rwth_lila100}{RGB}{122,111,172}
\definecolor{rwth_lila75}{RGB}{155,145,193}
\definecolor{rwth_lila50}{RGB}{188,181,215}
\definecolor{rwth_lila25}{RGB}{222,218,235}
\definecolor{rwth_lila10}{RGB}{242,240,247}

%
%---------------------------------------------------------------------------
%		Ergaenzungen / additions:
%
% 		Additions contain self-defined Tex and Latex commands as well
%		as a list of the words that cannot be separated.
%		(hyphenation)
% --------------------------------------------------------------------------
% --------------------------------------------------------------------------
% 		Einheiten / Units
% --------------------------------------------------------------------------
\DeclareSIUnit[]\kmh{\kilo\meter\per\hour}
\DeclareSIUnit[]\ms{\meter\per\second}
\DeclareSIUnit[]\mss{\meter\per\square\second}
\DeclareSIUnit[]\qm{\square\meter}

% --------------------------------------------------------------------------
% 	Eigene Befehle / Own commands
% --------------------------------------------------------------------------

% Differenzialoperator / Differential operator 
\newcommand*{\diff}{\mathop{}\!\mathrm{d}}

% --------------------------------------------------------------------------
% 		vorgegebene Trennung von Woertern / predefined seperation of words 
% --------------------------------------------------------------------------
\hyphenation{
ex-amp-le
}

% --------------------------------------------------------------------------
% 		Pakete / Packages 
% --------------------------------------------------------------------------

% Die folgenden Pakete wurden bereits in igm.sty / the following packages were already included in igm.sty

% {scrhack}	 				% Zusammenspiel von einigen Paketen mit KOMA-Script / Interaction of several packages with the KOMA-Script
% [utf8]{inputenc}    		% Input-Encodung / Input-encoding
% {babel}          			% Rechtschreibunterstuetzung / Spell aid
% {csquotes} 				% Deutsche Anfuehrungszeichen / German quotation marks
% [T1]{fontenc}         	% T1-kodierte Schriften, korrekte Trennmuster fuer Worte mit Umlauten / T1-encoded fonts, correct seperation of words with umlauts
% {lmodern}					% Schoenere Schrift / Nicer font
% {textcomp}				% Sonderzeichen im Text (z.B. €) / Special characters in the text (e.g. €)
% {textgreek}				% Griechische Symbole im Text / Greek symbols in the text
% {setspace}        		% Zeilenabstand / Line spacing
% {scrpage2}				% Kopf- und Fusszeilen / Header and footer
% {caption}       			% mehrzeilige Captions ausrichten / adjust multiline captions
% {booktabs}		 		% Schoene horizontale Linien / horizontal lines
% {multirow}		 		% Spalten und Zeilen weiter unterteilen / Divide lines and columns further
% {rccol}			 		% Ausrichtung von Spalten am Dezimalzeichen / Align columns according to the decimal point
% {graphicx, psfrag} 		% Zum Einbinden von Grafiken / Incorporation of graphics
% {subcaption}          	% Unterabbildungen / Sub-illustrations
% {amsmath}  				% Fuer erweiterte mathematische Konstrukte / for complex mathematic constructions
% {mathtools}				% Fuer Mathematikformeln: Indizes oben links / Mathematical formula: Indeces top left
% {mathrsfs,amssymb}		% Fuer Mathematikformeln: Symbole / Mathematical formula: Symbols
% {amsfonts}				% Fuer Mathematikformeln: Schriften / Mathematical formula: Fonts
% {bm}						% Fettschrift fuer Matrizen und Vektoren / Bold lettering for matrices and vectors
% {arydshln}				% Linien fuer Matrizen und Vektoren / lines for matrices and vectors
% {biblatex}				% Quellenangaben / Citations
% {acronym}					% For the list of abbreviations
% {siunitx}					% Für schöne Einheiten 
% {pdflscape}       		% Seiten im Querformat im PDF richtig anzeigen / Display pages properly in landscape mode
% {hyperref}				% PDF mit Hyperlinks
% {geometry}				% margins

% weitere Pakete können eingebunden werden / additional packages can be included

%\usepackage{pgfplots}						% Zum Erstellen von Vektorgrafiken / Creation of vector graphics
%\pgfplotsset{compat=1.13}						
%\usetikzlibrary{external,positioning,calc,decorations.markings,arrows,shapes,patterns}
%\tikzexternalize
%\newcommand{\includetikz}[1]{
%	\tikzsetnextfilename{Abbildungen/AbbildungenKompiliert/#1}%
%	\input{Abbildungen/AbbildungenTIKZ/#1.tikz}}%



%---------------------------------------------------------------------------
% 		Standard Aufbau / Structure 
%
%		- Deckblatt / Title page
%		- Schmutztitel / Half title
%		- Aufgabenstellung / Issue
%		- Eidesstattliche Erklärung / Statutory declaration
%		- (Vorwort) / (Preface)
%		- Inhaltsverzeichnis / Table of contents
%		- Abkuerzungsverzeichnis / List of abbreviations
%		- Formelzeichenverzeichnis / List of symbols
%		- Einleitung / Introduction
%		- Hauptkapitel / Main chapters
%		- Zusammenfassung / Summary
%		- Ausblick / Outlook
%		- Literaturverzeichnis / List of literature
%		- Abbildungsverzeichnis / List of illustrations
%		- Tabellenverzeichnis / List of tables
%		- Anhang / Appendix
%		
% --------------------------------------------------------------------------


% --------------------------------------------------------------------------
% 		Metadaten / Meta data 
% --------------------------------------------------------------------------
\author{Isabel Paredes}
\authordegreefront{}
\authordegreeback{B.Sc.}
\studentno{415723}

\type{Master Thesis}
\title{Cross-Compiling ROS 2 Humble to WebAssembly for the Development of a Web Browser Supported Robotics Environment}
\submissiondate{22 April 2023}
\supervisor{M.Sc. Markus Schmitz and M.Sc. Wolf Vollprecht}

% Text der Aufgabenstellung / text of the issue
% The issue will be inserted here after being drafted and provided 
% by the supervisor beforehand. The issue should contain a detailed list of 
% all work packages. It should not exceed one page and the version handed to 
% the students has to be signed by the professor.

\newcommand{\bulletini}{$\textcolor{bgColor}{000}$ \footnotesize$\bullet$ \normalsize \  }

\issuetext{
The accessibility of ROS 2 has been greatly improved as it now supports development on the most common platforms (Ubuntu, Windows and macOS), however, it is oftentimes desirable to share robotic applications without the need to set up a full installation in a local environment. As an example, consider an algorithm that solves a robotics task and is developed and published in a research paper. The authors of this paper would like that any reader should be able to replicate those results, but even with extensive installation instructions not everyone would be able to replicate the authors' setup identically. One possibility would be for the authors to share their work from a web browser to make it more accessible.\\ \\
There exist tools such as those developed by Robot Web Tools which allow for a limited interaction with ROS from the browser, but these tools rely on a connection to a system already running ROS. Other methods rely on a connection to a server to provide access to a ROS environment, for example, those using Docker containers. To solve this problem, the creation of a ROS 2 environment which runs entirely on a web browser is proposed. This solution is divided into three main components: \\
\bulletini the cross-compilation of ROS 2 packages to WebAssembly \\
\bulletini the development of a middleware to handle all ROS communications on the browser \\
\bulletini the deployment of a web-based platform where the users can interact with ROS 2 \\ \\
The scope of this thesis will be dedicated to setting up the ground work for a functioning ROS 2 environment on a web browser. In other words, the essential ROS 2 packages will be cross-compiled to WebAssembly, a middleware to replace the currently employed DDS (Data Distribution Service) will be created to handle communications, and a website where this ROS 2 environment can be accessed will be deployed.\\ \\
The outcome of this project will be tested based on its performance according to the following criteria:\\
\bulletini executability of ROS 2 nodes on the browser \\
\bulletini communication between the nodes \\
\bulletini shareability of workspaces and packages with other ROS developers \\
The results of the tests will be analyzed and evaluated depending on how closely they meet the given requirements. \\ \\
In summary, this project is to set the foundation for a web-based ROS 2 environment with the possibility of being extensible to
custom applications. Future developments could include: \\
\bulletini addition of specialized ROS 2 packages such as ``MoveIt2'' or custom packages \\
\bulletini adaptation of the Jupyter-ROS extensions from JupyterLab to JupyterLite \\
\bulletini optimization of browser performance and middleware solution \\
\bulletini streamlining the process of cross-compiling packages for other ROS 2distribution \\ \\
Planned workflow: \\
\bulletini Literature review and familiarization with ROS 2 and WebAssembly (1 Week) \\
\bulletini Compile, test and publish base packages and ROS 2 client libraries (3 Weeks) \\
\bulletini Development of a DDS replacement to handle communications (4 Weeks) \\ 
\bulletini Publish and subscribe to topics and define message types (3 Weeks) \\
\bulletini File management on the browser (1 Week) \\
\bulletini Compile example packages and packages for robot visualization (2 Weeks) \\
\bulletini Testing performance, establishing benchmarks, and comparison to other tools (2 Weeks) \\
\bulletini Testing shareability and deployment with a group of ROS developers (1 Week) \\
\bulletini Analysis of test results (1 Week) \\
\bulletini Writing of the report (2 Week)   
}

% --------------------------------------------------------------------------
% 		Literaturdatei / Bibliography file
% --------------------------------------------------------------------------
\addbibresource{04_refer/References.bib} 

\begin{document}
\pagecolor{bgColor}
\color{textColor}

\frontmatter	% Beginn des Vorspanns / Begin of the prefix

% --------------------------------------------------------------------------
% 		Deckblatt / Title page
% --------------------------------------------------------------------------
\TitlePageIGMR     

% --------------------------------------------------------------------------
% 		Aufgabenstellung / Issue
% --------------------------------------------------------------------------
\IssueIGMR

\ifoot[]{}
\cfoot[]{}
\ofoot[]{}

% --------------------------------------------------------------------------
% 		Eidesstattliche Erklärung / Statutory declaration
% --------------------------------------------------------------------------
\DeclarationIGMR

% --------------------------------------------------------------------------
%		Vorwort / Preface (optional)
% -------------------------------------------------------------------------- 
%\ihead[]{\multilang{Vorwort}{Preface}}
\chead[]{}
\ohead[]{\pagemark}

\textbf{\multilang{Vorwort}{Preface}}
\vspace{0.5cm}

\multilang{Hier kann ein Vorwort eingefügt werden.}{A preface can be inserted here.}

\vspace{1.5cm}

Aachen, \Msubmissiondate

% --------------------------------------------------------------------------<
%		Inhaltsverzeichnis / Table of contents
% -------------------------------------------------------------------------- 

\ihead[]{\headmark}
\pdfbookmark[1]{\contentsname}{toc}
\tableofcontents					% Inhaltsverzeichnis / Table of contents

% --------------------------------------------------------------------------
%		Formelzeichenverzeichnis / List of symbols 
% -------------------------------------------------------------------------- 
% Formula symbols

\newcommand{\acrounit}[1]{
\acroextra{\makebox[18mm][l]{\si{#1}}}
}

\chapter*{\multilang{Formelzeichen und Indizes}{Formula symbols and indices}}

% Headmarks need to be enforced in every chapter*
\ihead[]{\multilang{Formelzeichen und Indizes}{Formula symbols and indices}}

% A list of all content has to be enforced in every chapter*
\addcontentsline{toc}{chapter}{\multilang{Formelzeichen und Indizes}{Formula symbols and indices}}

\section*{\multilang{Formelzeichen aus lateinischen Kleinbuchstaben}{Lower case latin letters as formula symbols}}
\begin{acronym}[LONGEST]
\acro{a}[\ensuremath{a}]{\acrounit{\mss}acceleration}
\end{acronym}

\section*{\multilang{Formelzeichen aus lateinischen Großbuchstaben}{Upper case latin letters as formula symbols}}
\begin{acronym}[LONGEST]
\acro{A}[\ensuremath{A}]{\acrounit{\qm}surface}
\end{acronym}

\section*{\multilang{Formelzeichen aus griechischen Kleinbuchstaben}{Lower case greek letters as formula symbols}}
\begin{acronym}[LONGEST]
\acro{alpha}[\ensuremath{\alpha}]{\acrounit{-}Weighting factor}
\end{acronym}

\section*{\multilang{Formelzeichen aus griechischen Großbuchstaben}{Upper case greek letters as formula symbols}}
\begin{acronym}[LONGEST]
\acro{Omega}[\ensuremath{\Omega}]{\acrounit{\radian\per\meter} Angular frequency}
\end{acronym}

\section*{\multilang{Indizes}{Indices}}
\begin{acronym}[LONGEST]
\acro{in_a}[\ensuremath{a}]{Amplitude}
\end{acronym}
			%  Formelzeichenverzeichnis / List of symbols

% --------------------------------------------------------------------------
%		Abkuerzungsverzeichnis / List of abbreviations
% -------------------------------------------------------------------------- 
\chapter*{\multilang{Abkürzungsverzeichnis}{List of abbreviations}}

% Headmarks need to be enforced in every chapter*
\ihead[]{\multilang{Abkürzungsverzeichnis}{List of abbreviations}}

% A list of all content has to be enforced in every chapter*
\addcontentsline{toc}{chapter}{\multilang{Abkürzungsverzeichnis}{List of abbreviations}}

\section*{\multilang{Allgemeine Abkürzungen}{General abbreviations}}
\begin{acronym}[LONGEST]

    \acro{API}[API]{Application Programming Interface}
    \acro{D3wasm}[D3wasm]{Doom 3 WebAssembly}
    \acro{DCPS}[DCPS]{Data-Centric Publish Subscribe}
    \acro{DDS}[DDS]{Data-Distribution Service}
    \acro{emsdk}[emsdk]{Emscripten SDK}
    \acro{GUI}[GUI]{Graphical User Interface}
    \acro{HTML}[HTML]{HyperText Markup Language}
    \acro{IDL}[IDL]{Interface Description Language}
    \acro{JSON}[JSON]{JavaScript Object Notation}
    \acro{LED}[LED]{Light Emitting Diode}
    \acro{LIFO}[LIFO]{Last In, First Out}
    \acro{MVP}[MVP]{Minimum Viable Product}
    \acro{OMG}[OMG]{Object Management Group}
    \acro{QoS}[QoS]{Quality of Service}
    \acro{RMW}[RMW]{ROS Middleware}
    \acro{ROS}[ROS]{Robot Operating System}
    \acro{RTI}[RTI]{Real-Time Innovations}
    \acro{RTPS}[RTPS]{Real-Time Publish Subscribe Protocol}
    \acro{SDK}[SDK]{Software Development Kit}
    \acro{TCPROS}[TCPROS]{Transmission Control Protocol ROS}
    \acro{UI}[UI]{User Interface}
    \acro{URDF}[URDF]{Universal Robotic Description Format}
    \acro{VCS}[VCS]{Version Control System}
    \acro{VM}[VM]{Virtual Machine}
    \acro{WASM}[WASM]{Web Assembly}
    \acro{WebDDS}[WebDDS]{Web-Enabled DDS}
    \acro{YAML}[YAML]{YAML Ain't Markup Language}

\end{acronym}

	% Abkuerzungsverzeichnis / List of abbreviations

% --------------------------------------------------------------------------
% 		Inhalt / Contents
% --------------------------------------------------------------------------

\mainmatter		% Beginn des Hauptteils / Begin of the main body
\ihead[]{\headmark}

%%%%%%%%%%%%%%%%%%%%%%%%%%%%%%%%%%%%%%%%%%%%%%%%%%%%%%%%%%%%%%%%%%%%%%%%%%%%
%%%%%%%%%%%%%%%%%%%%%%%%%%%%%%%%%%%%%%%%%%%%%%%%%%%%%%%%%%%%%%%%%%%%%%%%%%%%
%%%%%%%%%%%%%%%%%%%%%%%%%%%%%%%%%%%%%%%%%%%%%%%%%%%%%%%%%%%%%%%%%%%%%%%%%%%%
% \include{Contents/01-Introduction}

% \chapter{Kapitel 2}
\label{cha:kap2}


% --------------------------------------------------------------------------
% 		Abschnitt 2.1
% --------------------------------------------------------------------------
\section{Abschnitt 1}
\label{sec:kap2ab1}


% --------------------------------------------------------------------------
% 		Abschnitt 2.2
% --------------------------------------------------------------------------
\section{Abschnitt 2}
\label{sec:kap2ab2}

 

% \chapter{Sample chapter}
\label{cha:samplechapter}
This sample chapter serves as an illustration of frequently used elements in LaTeX such as enumerations, illustrations, tables or equations. It makes no claim to completeness and can gladly be extended. A detailed description of all commands used in LaTex can be found in the list of commands.
The samples can be copied and then adjusted.  
% Section 1 --------------------------------------------------------------
% 		Name of the section
% --------------------------------------------------------------------------
\section{Enumerations}
\label{sec:enumerations}

\subsection{Enumerations using dots}
\label{enumeration_dots}

\begin{itemize}
	\item body
	\item connecting elements
	\item coupling elements
\end{itemize}

\subsection{Enumerations using numbers}
\label{enumeration_numbers}

\begin{enumerate}
	\item body
	\item connecting elements
	\item coupling elements
\end{enumerate}

\cleardoublepage
% Section 2 --------------------------------------------------------------
% 		Name of the section
% --------------------------------------------------------------------------
\section{Illustrations}
\label{sec:illustrations}

\begin{figure}[htbp]
	\centering
		\includegraphics[width = 0.5\textwidth]{Contents/Resources/superc.jpeg}
	\caption[Image (short caption without source)]{Image (detailed caption including source, Source: \cite[1]{Sample.2012})}
	\label{fig:a_image}
\end{figure}

\begin{figure}[htbp]
	\centering
	\begin{subfigure}[t]{0.46\textwidth}
		\includegraphics[width = 1\textwidth]{Contents/Resources/superc.jpeg}
		\caption{Image 1}
		\label{fig:image1}
	\end{subfigure}
	\begin{subfigure}[t]{0.46\textwidth}
		\includegraphics[width = 1\textwidth]{Contents/Resources/superc.jpeg}
		\caption{Image 2}
		\label{fig:image2}
	\end{subfigure}
	\caption[Two images]{Image 1 (a) and Image 2 (b)}
	\label{fig:multiple_images}
\end{figure}

\cleardoublepage
% Section 3 --------------------------------------------------------------
% 		Name of the section
% --------------------------------------------------------------------------
\section{Tables}
\label{sec:tables}

\begin{table}[htbp]
	\centering	
	\caption{Tables with automatic alignment}
		\begin{tabular}{lcr}
	 	\toprule
	 	l & c & r\\
	 	\midrule
		a & b & c\\[0.25em]
		aa & bb & cc\\[0.25em]
		aaa & bbb & ccc\\
		\bottomrule
	\end{tabular}	
	\label{tab:table1}
\end{table}

\begin{table}[htbp]
  \centering
  \caption{Tables aligned to seperators}
    \begin{tabular}{R{4}{3} R{4}{0}}
    \toprule
          \multicolumn{1}{c}{a} & \multicolumn{1}{c}{b}\\
    \midrule
	1,234 & 1234\\
	12,34 & 123\\
	123,4 & 12\\
	1234  & 1\\
    \bottomrule
    \end{tabular}
  \label{tab:table2}
\end{table}

\begin{table}[htbp]
	\centering	
	\caption{Table with multiple cells across various rows and columns}
		\begin{tabular}{lcr}
	 	\toprule
	 	l & c & r\\
	 	\midrule
		\multicolumn{2}{c}{ab} & c\\[0.25em]
		\multirow{2}{*}{aa} & bb & cc\\[0.25em]
		& bbb & ccc\\
		\bottomrule
	\end{tabular}	
	\label{tab:table3}
\end{table}

%\begin{table}[htbp]
%  \centering
%  \caption{multiple sub-tables}
%  \subtable[table 1]{
%    \centering  
%\begin{tabular}{lcr}
%	 	\toprule
%	 	l & c & r\\
%	 	\midrule
%		a & b & c\\[0.25em]
%		aa & bb & cc\\[0.25em]
%		aaa & bbb & ccc\\
%		\bottomrule
%	\end{tabular}	
%  }
%  \subtable[Tabelle 2]{
%    \centering  
%\begin{tabular}{lcr}
%	 	\toprule
%	 	l & c & r\\
%	 	\midrule
%		a & b & c\\[0.25em]
%		aa & bb & cc\\[0.25em]
%		aaa & bbb & ccc\\
%		\bottomrule
%	\end{tabular}	
%  }
%\label{tab:tables}
%\end{table}

\cleardoublepage
% Section 4 --------------------------------------------------------------
% 		Name of the section
% --------------------------------------------------------------------------
\section{Equations}
\label{sec:equations}

\begin{align}
	F = m a 
	\label{eqn:newton_en}
\end{align}


\section{Citation options}

Cite a source: \cite{Sample.2012}\\
Cite a source with a page reference: \cite[12-16]{Sample.2012}\\
Cite multiple Sources: \cites{Samplem.2012}{Samplef.2011}\\
Cite multiple sources with page references: \cites[12-16]{Samplem.2012}[3]{Samplef.2011}\\


% \chapter{Beispielkapitel}
\label{cha:beispielkapitel}
Dieses Beispielkapitel dient der Darstellung häufig verwendeter Elemente in LaTeX wie Aufzählungen, Abbildungen, Tabellen oder Gleichungen. Es hat keinen Anspruch auf Vollständigkeit und kann gerne erweitert werden. Eine ausführliche Beschreibung der LaTeX-Befehle kann der Befehlsübersicht entnommen werden.
Die  Beispiele können kopiert und dann angepasst werden.  
% Abschnitt 1 --------------------------------------------------------------
% 		Name des Abschnittes
% --------------------------------------------------------------------------
\section{Aufzählungen}
\label{sec:aufzaehlungen}

\subsection{Aufzählungen mit Punkten}
\label{aufzaehlung_punkte}

\begin{itemize}
	\item Körper
	\item Bindungselemente
	\item Koppelelemente
\end{itemize}

\subsection{Aufzählungen mit Zahlen}
\label{aufzaehlung_zaheln}

\begin{enumerate}
	\item Körper
	\item Bindungselemente
	\item Koppelelemente
\end{enumerate}

\cleardoublepage
% Abschnitt 2 --------------------------------------------------------------
% 		Name des Abschnittes
% --------------------------------------------------------------------------
\section{Abbildungen}
\label{sec:abbildungen}

\begin{figure}[htbp]
	\centering
		\includegraphics[width = 0.5\textwidth]{Contents/Resources/superc.jpeg}
	\caption[Eine Abbildung (kurze Abbildungsunterschrift ohne Quelle)]{Eine Abbildung (lange Abbildungsunterschrift mit Quelle, Quelle: \cite[1]{Mustermann.2012})}
	\label{fig:eine_abbildung}
\end{figure}

\begin{figure}[htbp]
	\centering
	\begin{subfigure}[t]{0.46\textwidth}
		\includegraphics[width = 1\textwidth]{Contents/Resources/superc.jpeg}
		\caption{Bild 1}
		\label{fig:bild1}
	\end{subfigure}
	\begin{subfigure}[t]{0.46\textwidth}
		\includegraphics[width = 1\textwidth]{Contents/Resources/superc.jpeg}
		\caption{Bild 2}
		\label{fig:bild2}
	\end{subfigure}
	\caption[Zwei Abbildungen]{Bild 1 (a) und Bild 2 (b)}
	\label{fig:mehrere_abbildungen}
\end{figure}

\cleardoublepage
% Abschnitt 3 --------------------------------------------------------------
% 		Name des Abschnittes
% --------------------------------------------------------------------------
\section{Tabellen}
\label{sec:tabellen}

\begin{table}[htbp]
	\centering	
	\caption{Tabelle mit automatischer Ausrichtung}
		\begin{tabular}{lcr}
	 	\toprule
	 	l & c & r\\
	 	\midrule
		a & b & c\\[0.25em]
		aa & bb & cc\\[0.25em]
		aaa & bbb & ccc\\
		\bottomrule
	\end{tabular}	
	\label{tab:tabelle1}
\end{table}

\begin{table}[htbp]
  \centering
  \caption{Tabelle mit Ausrichtung an Trennungszeichen}
    \begin{tabular}{R{4}{3} R{4}{0}}
    \toprule
          \multicolumn{1}{c}{a} & \multicolumn{1}{c}{b}\\
    \midrule
	1,234 & 1234\\
	12,34 & 123\\
	123,4 & 12\\
	1234  & 1\\
    \bottomrule
    \end{tabular}
  \label{tab:tabelle2}
\end{table}

\begin{table}[htbp]
	\centering	
	\caption{Tabelle mit Zellen über mehrere Zeilen oder Spalten}
		\begin{tabular}{lcr}
	 	\toprule
	 	l & c & r\\
	 	\midrule
		\multicolumn{2}{c}{ab} & c\\[0.25em]
		\multirow{2}{*}{aa} & bb & cc\\[0.25em]
		& bbb & ccc\\
		\bottomrule
	\end{tabular}	
	\label{tab:tabelle3}
\end{table}

%\begin{table}[htbp]
%  \centering
%  \caption{Mehrere Untertabellen}
%  \subtable[Tabelle 1]{
%    \centering  
%\begin{tabular}{lcr}
%	 	\toprule
%	 	l & c & r\\
%	 	\midrule
%		a & b & c\\[0.25em]
%		aa & bb & cc\\[0.25em]
%		aaa & bbb & ccc\\
%		\bottomrule
%	\end{tabular}	
%  }
%  \subtable[Tabelle 2]{
%    \centering  
%\begin{tabular}{lcr}
%	 	\toprule
%	 	l & c & r\\
%	 	\midrule
%		a & b & c\\[0.25em]
%		aa & bb & cc\\[0.25em]
%		aaa & bbb & ccc\\
%		\bottomrule
%	\end{tabular}	
%  }
%\label{tab:tabellen}
%\end{table}

\cleardoublepage
% Abschnitt 4 --------------------------------------------------------------
% 		Name des Abschnittes
% --------------------------------------------------------------------------
\section{Gleichungen}
\label{sec:gleichungen}

\begin{align}
	F = m a 
	\label{eqn:newton}
\end{align}


\section{Anführungszeichen}
\label{sec:anfuehrungszeichen}
Es gibt mehrere Möglichkeiten deutsche Anführungszeichen einzufügen:\\
\glqq test\grqq\\
"`test"'\\

\section{Zitationen}

Zitation einer Quelle: \cite{Mustermann.2012}\\
Zitation einer Quelle mit Seitenangabe: \cite[12-16]{Mustermann.2012}\\
Zitation mehrerer Quellen: \cites{Mustermann.2012}{Musterfrau.2011}\\
Zitation mehrerer Quellen mit Seitenangabe: \cites[12-16]{Mustermann.2012}[3]{Musterfrau.2011}\\



% \include{Contents/98-Summary}
%
% \chapter{Outlook}\label{cha:outlook}


%       EXTRAS
%		x Title page
%		- Half title??
%		- Issue [last]
%		x Statutory declaration
%		- (Preface)
%		x Table of contents
%		- List of abbreviations
%		- List of symbols

%       MAIN CHAPTERS
%		- Introduction
%		- Chapters
%		- Conclusion

%       EXTRAS
%		- List of literature
%		- List of illustrations
%		- List of tables
%		- Appendix

\chapter{Introduction}\label{cha:introduction}

\section{Robot Operating System 2}

    \ac{ROS} 2

\section{Motivation}


    \ac{WASM} 

\chapter{Literature Review}\label{cha:literature}


\section{ROS on Web}\label{sec:ros_on_web}

\section{ROSbridge}\label{sec:rosbridge}

\section{ROS Control Center}\label{sec:roscontrol}

\section{ROSboard}\label{sec:rosboard}

\section{ROSlink}\label{sec:roslink}

\section{Foxglove Studio}\label{sec:foxglove}

\section{Unity WASM}

unreal on the web

\chapter{Concept Realization}\label{cha:concept}


\section{Concept}\label{sec:concept}

Ideal scenario: 
- click on a link and run ROS
- connect to a robot via bluetooth
- share simulations and algorithms

\section{Technical Levels}\label{sec:levels}

    \subsection{User Levels of Interaction}\label{sub:user_levels}

    \subsection{Technical Levels of Implementation}\label{sub:tech_levels}

\section{Scope}\label{sec:scope}

    - Middleware replacement (why sockets don't work)

    - JavaScript ``ROS master''


% Too many sections
% Add a note to point or link to issues about the changes
% Thesis will come with a git repo
% Add QR codes to link to specific commit for each step of development
% 1.0 is ready to be published
% Add a note about submitting to ROSCon
% Submit proposal in March

\chapter{Methodology}\label{cha:methodology}

- Development environment
- Building tools
- Testing tools (chrome, firefox)


\section{Development Environment}\label{sec:devenv}


\section{Cross-Compilation Tools}\label{sec:cross}


\section{Testing Environment}\label{sec:testing}



% New stuff
\section{Package Building Process}\label{cha:build}

- Emscripten
- Colcon
- Toolchains

\section{Environment}


\section{Tools}


\section{Post Processing}


\chapter{Middleware}

    A significant change from \ac{ROS} 1 to \ac{ROS} 2 is the shift from a custom transport layer consisting of \ac{TCPROS} to \ac{DDS}. \ac{DDS} is a publish-subscribe communication standard defined by \ac{OMG}. \ac{DDS} uses \ac{IDL} for defining and serializing messages~\cite{rosondds}. In contrast to \ac{ROS} 1, which requires a \ac{ROS} master in order for nodes to discover and communicate with each other, \ac{ROS} 2 discovery system is handled by \ac{DDS} and each of the \ac{DDS} vendors provides different options for customizing the communication layer.

    One notable advantage of moving away from a custom transport protocol is that the \ac{ROS} client libraries are now agnostic to the middleware interface; this means that the complexities of the \ac{DDS} implementation are not exposed to the end user~\cite{ros2middle}. As a consequence, multiple middleware interfaces can be implemented as long as they fulfill the following requirements: 
    \begin{itemize}
        \item publishing and subscribing
        \item message serialization
        \item discovery
    \end{itemize}
    
    The interaction between the \ac{ROS} user, the \ac{ROS} client libraries, and the middleware layers is shown in Figure~\ref{fig:middleware}.

    \begin{figure}[htbp]
        \centering
        \vspace{1em}
        \begin{tikzpicture}
            \node (user) [packBox] {ROS User};

            \node (rcl) [packBox, yshift=-2cm] {ROS Client Libraries \\ \small\textsf{rclcpp | rclpy}};

            \node (interface) [packBox, yshift=-4cm] {Middleware Interface \\ \small\textsf{rmw}};

            \node (implementation) [packBox, yshift=-6cm] {\textbf{Middleware Implementation} \\ \small\textsf{fastrtps | cyclonedds | connextdds | gurumdds | custom}};

            \begin{scope}[transform canvas={xshift=-0.5cm}]
                \draw [-to] (user) -- (rcl);
                \draw [-to] (rcl) -- (interface);
                \draw [-to] (interface) -- (implementation);
            \end{scope}

            \begin{scope}[transform canvas={xshift=0.5cm}]
                \draw [to-] (user) -- (rcl);
                \draw [to-] (rcl) -- (interface);
                \draw [to-] (interface) -- (implementation);
            \end{scope}

        \end{tikzpicture}
        \vspace{1em}
        \caption{Relations between the user, the \ac{ROS} client libraries and the middleware packages~\cite{ros2middle}.}
        \label{fig:middleware}
    \end{figure}


\section{Supported Implementations}

    Currently, \ac{ROS} 2 releases provide full support for three middleware implementations: eProsima Fast \ac{DDS}, Eclipse Cyclone \ac{DDS}, and \ac{RTI} Connext \ac{DDS}. The binaries also support Gurum \ac{DDS}, but the implementation requires a separate installation~\cite{docsdds}. 

    \subsection{eProsima Fast DDS}

    eProsima Fast \ac{DDS}, also known as Fast \ac{RTPS}, is the default middleware implementation for \ac{ROS} 2 packages. Some of the main advantages of Fast \ac{DDS} is that it is free, open source, and it is developed for most platforms including Linux, Windows, Mac OS, and QNX. A rich set of \ac{QoS} parameters is also available for tuning the communication protocols to any particular system. Fast \ac{DDS} follows a \ac{DCPS} model, which consists four elements: publishers, subscribers, topics, and domains~\cite{dcps}. This model introduces the concept of \textsf{Data Writers} and \textsf{Data Readers} which, as the names imply, have read and write permissions to the ``Global Data Space'' as specified by the \ac{DDS} standard~\cite{introdds}. Figure~\ref{fig:ddsdomain} displays an example of the Fast \ac{DDS} architecture and demonstrates how the different elements interact with each other.

    \begin{figure}[htbp]
        \centering
        \vspace{1em}
        \begin{tikzpicture}
            \node (domain) [
                box, 
                minimum width=14cm,
                text depth=6cm,
                fill=igmrLightBlue!10!bgColor,
            ] {DDS Domain};
            
            \node (p1) [
                box,
                xshift=-4cm,
                minimum width = 4cm,
                text depth=4.5cm,
                fill=igmrLightBlue!40!bgColor,
            ] {Domain Participant};

            \node (pub1) [
                box,
                xshift=-4cm,
                yshift=0.5cm,
                minimum width=3.5cm,
                text depth=1cm
            ] {Publisher};

            \node (dw1) [
                box,
                xshift=-4cm,
                yshift=0.2cm,
                minimum width=3cm,
                fill=bgColor,
            ] {Data Writer};

            \node (sub1) [
                box,
                xshift=-4cm,
                yshift=-1.5cm,
                minimum width=3.5cm,
                text depth=1cm
            ] {Subscriber};

            \node (dr1) [
                box,
                xshift=-4cm,
                yshift=-1.8cm,
                minimum width=3cm,
                fill=bgColor,
            ] {Data Reader};

            \node (p2) [
                box,
                xshift=4cm,
                minimum width=4cm,
                text depth=4.5cm,
                fill=igmrLightBlue!40!bgColor,
            ] {Domain Participant};

            \node (sub2) [
                box,
                xshift=4cm,
                yshift=0.5cm,
                minimum width=3.5cm,
                text depth=1cm
            ] {Subscriber};

            \node (dr2) [
                box,
                xshift=4cm,
                yshift=0.2cm,
                minimum width=3cm,
                fill=bgColor,
            ] {Data Reader};

            \node (t1) [
                box,
                minimum width=2cm,
                minimum height=1cm,
                fill=igmrLightBlue!40!bgColor,
            ] {Topic};

            \node (dot1) [
                rectangle,
                xshift=-1.5cm,
            ] {};

            \draw [thick] (t1) -- (dot1.center);
            \draw [arrow] (dw1.east) -- (dw1.east-|t1.west);
            \draw [arrow] (dot1.center) |- (dr1);
            \draw [arrow] (t1.east|-dr2.west) -- (dr2.west);

        \end{tikzpicture}
        \vspace{1em}
        \caption{Instance of a typical Fast \ac{DDS} domain model.}
        \label{fig:ddsdomain}
    \end{figure}

    \subsection{Eclipse Cyclone DDS}

        Similar to Fast \ac{DDS}, Cyclone \ac{DDS} is free and open source and supports the three major platforms, Linux, Windows, and Mac OS. Cyclone \ac{DDS} offers a ``data-centric'' architecture with space- and time-decoupling with a zero configuration discovery system~\cite{eclipse}. Additionally, this implementation includes Python bindings to simplify the definition of data types.

    \subsection{RTI Connext DDS}

        \ac{RTI}

    \subsection{GurumNetworks Gurum DDS}

    


\section{Custom Middleware}

    Why it needs to be replaced

    \subsection{Email}

    \subsection{Zenoh}

    Minimal implementation (minimal set of functions)

    
\section{Substituting ROS 2 Middleware}

    At run time

    At build time

\section{Custom Middleware Design}

    Design of middleware packages (tree diagram or something)

    \subsection{\textsf{wasm\_cpp}}


    \begin{figure}[htbp]
        \centering
        \begin{tikzpicture}%[show background grid]
            \begin{abstractclass}[text width=5cm]{Participant}{0,0}
                \attribute{- name : String}
                \attribute{- role : String}
                \attribute{- gid  : String}

                \operation{- is\_valid\_name()}
                \operation{- is\_valid\_role()}
                \operation{- registration()}
                \operation{- deregistration()}
            \end{abstractclass}

            \begin{class}[text width=5cm]{Publisher}{-5,-6}
                \inherit{Participant}
                \attribute{- name = topic\_name}
                \attribute{- role = publisher}
                \operation{+ publish(message : String)}
            \end{class}

            \begin{class}[text width=5cm]{Subscriber}{5,-6}
                \inherit{Participant}
                \attribute{- name = topic\_name}
                \attribute{- role = subscriber}
                \operation{+ get\_message() : String}
            \end{class}

            \begin{class}[text width=6.5cm]{ServiceServer}{-4,-11}
                \inherit{Participant}
                \attribute{- name = service\_name}
                \attribute{- role = service\_server}
                \operation{+ take\_request() : String}
                \operation{+ send\_response(response : String)}
            \end{class}

            \composition{ServiceServer}{}{}{Publisher}
            \composition{ServiceServer}{}{}{Subscriber}

            \begin{class}[text width=6.5cm]{ServiceClient}{4,-11}
                \inherit{Participant}
                \attribute{- name = service\_name}
                \attribute{- role = service\_client}
                \operation{+ send\_request(request : String)}
                \operation{+ take\_response() : String}
                \operation{+ is\_service\_available() : Bool}
            \end{class}

            \composition{ServiceClient}{}{}{Publisher}
            \composition{ServiceClient}{}{}{Subscriber}

        \end{tikzpicture}
        \caption{A class diagram}
    \end{figure}
    


\chapter{ROS 2 Middleware Implementation for WebAssembly}

    The design of a custom middleware implementation, \textsf{rms\_wasm}, that can be cross-compiled to WebAssembly modules is divided into three distinct packages as observed in Figure~\ref{fig:rmwwasm}.

    \begin{figure}[htbp]
        \centering
        \vspace{1em}
        \begin{tikzpicture}
            \node (pkg) [
                rectangle,
                rounded corners,
                draw = textColor,
                fill = igmrLightBlue!10!bgColor,
                text height = 7cm,
                align = justify,
                minimum width = 14cm,
                text width = 13.5cm,
            ] {\textsf{rmw\_wasm}};

            \node (rwc) [
                packBox,
                yshift = 2.5cm,
                fill = igmrLightBlue!30!bgColor,
                ] {\textsf{rmw\_wasm\_cpp}};

            \node (wcpp) [
                packBox,
                yshift = 0.5cm,
                fill = igmrLightBlue!30!bgColor,
                ] {\textsf{wasm\_cpp}};

            \node (wjs) [
                packBox,
                yshift = -2.5cm,
                fill = igmrLightBlue!30!bgColor,
                ] {\textsf{wasm\_js}};
            
            \node (ems) [
                rectangle,
                yshift = -1cm,
                xshift = 1.5cm,
                ] {\textsf{emscripten::val}};

            \draw[thick] (rwc) -- (wcpp);
            \draw[thick] (wcpp) -- (wjs);

        \end{tikzpicture}
        \vspace{1em}
        \caption{Architecture of custom middleware implementation to target WebAssembly.}
        \label{fig:rmwwasm}
    \end{figure}

    \section{\textsf{rmw\_wasm\_cpp}}

        The first package, \textsf{rmw\_wasm\_cpp}, works as the ``adapter'' between \ac{ROS} 2 and the designed middleware. This package implements all of the functions required for \textsf{rmw} as described in Section~\ref{ssec:minimal}. The source code for this package is entirely written in C++ and can be found: TODO:

        TODO: add diagram

    \section{\textsf{wasm\_cpp}}

        The role of \textsf{wasm\_cpp} is to TODO: and to function as a bridge to JavaScript functions.
    \section{\textsf{wasm\_js}}


    \begin{figure}[htbp]
        \centering
        \begin{tikzpicture}%[show background grid]
            \begin{abstractclass}[text width=5cm]{Participant}{0,0}
                \attribute{- name : String}
                \attribute{- role : String}
                \attribute{- gid  : String}

                \operation{- is\_valid\_name()}
                \operation{- is\_valid\_role()}
                \operation{- registration()}
                \operation{- deregistration()}
            \end{abstractclass}

            \begin{class}[text width=5cm]{Publisher}{-5,-6}
                \inherit{Participant}
                \attribute{- name = topic\_name}
                \attribute{- role = publisher}
                \operation{+ publish(message : String)}
            \end{class}

            \begin{class}[text width=5cm]{Subscriber}{5,-6}
                \inherit{Participant}
                \attribute{- name = topic\_name}
                \attribute{- role = subscriber}
                \operation{+ get\_message() : String}
            \end{class}

            \begin{class}[text width=6.5cm]{ServiceServer}{-4,-11}
                \inherit{Participant}
                \attribute{- name = service\_name}
                \attribute{- role = service\_server}
                \operation{+ take\_request() : String}
                \operation{+ send\_response(response : String)}
            \end{class}

            \composition{ServiceServer}{}{}{Publisher}
            \composition{ServiceServer}{}{}{Subscriber}

            \begin{class}[text width=6.5cm]{ServiceClient}{4,-11}
                \inherit{Participant}
                \attribute{- name = service\_name}
                \attribute{- role = service\_client}
                \operation{+ send\_request(request : String)}
                \operation{+ take\_response() : String}
                \operation{+ is\_service\_available() : Bool}
            \end{class}

            \composition{ServiceClient}{}{}{Publisher}
            \composition{ServiceClient}{}{}{Subscriber}

        \end{tikzpicture}
        \caption{A class diagram}
    \end{figure}
    


\section{Design of Web Elements}

\section{Web Workers}

    

\section{Message Stacks}

    \begin{figure}[htbp]
        \centering
        \begin{subfigure}[t]{0.32\textwidth}
            \includegraphics[height=0.9\textwidth]{07_stack0.png}
            \caption{Filling stack}
        \end{subfigure}
        \begin{subfigure}[t]{0.32\textwidth}
            \includegraphics[height=0.9\textwidth]{07_stack1.png}
            \caption{Overwriting stack}
        \end{subfigure}
        \begin{subfigure}[t]{0.32\textwidth}
            \includegraphics[height=0.9\textwidth]{07_stack2.png}
            \caption{Reading stack}
        \end{subfigure}
        \caption{Modified Circular Stack, \ac{LIFO}}\label{fig:circleStack}
    \end{figure}

- Web workers, what are they? why are they needed?
- Communication channels
- Registry of topics/subs/pubs
- Message handling
width

\chapter{Concept Assessment}\label{cha:assessment}

- Survey
- Performance measures
- Limitations

\chapter{Discussion}

This project set off to demonstrate the potential of running \ac{ROS} 2 on the browser by using WebAssembly. The proposed solution was subdivided into three major components: the creation of a custom middleware implementation, a method for cross-compiling \ac{ROS} 2 packages to WebAssembly, and the deployment of a web-based platform to offer users an opportunity to interact with \ac{ROS}.

The design of a custom middleware implementation, \textsf{rmw-wasm}, has been described in Chapter~\ref{cha:rmw}. \textsf{rmw-wasm} is composed of three distinct packages: \textsf{rmw-wasm-cpp}, \textsf{wasm-cpp}, and \textsf{wasm-cpp}. The interface between \ac{ROS} and the custom middleware implementation is carried out by \textsf{rmw-wasm-cpp}. The bridge between C++ and JavaScript is managed by \textsf{wasm-cpp}. And \textsf{wasm-js} handles the traffic of all \ac{ROS} participants and communications on the browser.

Two methods were developed to simplify the cross-compilation of \ac{ROS} 2 packages to WebAssembly. The simplest of the two involves using the GitHub workflow given in Appendix~\ref{sec:apxworkflow}. Running the GitHub workflow does not require the installation of any tools locally. The only optional requirement for the workflow is to provide a package to be cross-compiled; however, if the target package is already included in the core \ac{ROS} packages, then providing the name of the package as an input to the workflow is sufficient. The second method for building packages is more complicated because it requires a local setup. This second method uses the \textsf{blasm.sh} script given in Appendix~\ref{sec:apxblasm} but in order for the script to run properly, Emscripten, the \textsf{colcon} tools, CMake, Python, and a few additional dependencies must be pre-installed in the build environment. The procedure for building packages locally follows the same steps as those listed in the GitHub workflow, the only difference is that the workflow is automated.

A website was deployed to make the demonstrations described in Chapter~\ref{cha:assessment} available to the general public. The website can be accessed through this link: \href{https://ros2wasm.dev/}{https://ros2wasm.dev/}. The provided demonstrations range in user interaction level, from non-interactive ($\mathcal{I}1$) to intermediate ($\mathcal{I}4$), and from complexity level $\mathcal{C}1$ through $\mathcal{C}5$. The demonstrations show that multiple \ac{ROS} nodes can be run on the browser and that these nodes are capable of communicating with each other. A website running a particular set of nodes can easily be shared with other users with a link, however, the process of setting up such website has not yet been simplified enough for beginners. 

In summary, this project accomplish the goal of setting the foundation for a web-based \ac{ROS} 2 environment. The work completed made it possible for any user, from beginner to expert, to access a website and run \ac{ROS} nodes on the browser without any need to install any packages or have any knowledge of \ac{ROS} in the first place. It is also clear that more developments will need to be made, starting with the incomplete levels $\mathcal{C}6-\mathcal{C}9$ and $\mathcal{I}5-\mathcal{I}6$, before this project becomes a viable alternative to a native \ac{ROS} 2 installation.


\section{Outlook}

- JupyterLite
- client library
- Zethus and visualizations
- Message conversion
- Compiling on the browser
- Packaging Gazebo
- Optimization
- actions



%%%%%%%%%%%%%%%%%%%%%%%%%%%%%%%%%%%%%%%%%%%%%%%%%%%%%%%%%%%%%%%%%%%%%%%%%%%%
%%%%%%%%%%%%%%%%%%%%%%%%%%%%%%%%%%%%%%%%%%%%%%%%%%%%%%%%%%%%%%%%%%%%%%%%%%%%
%%%%%%%%%%%%%%%%%%%%%%%%%%%%%%%%%%%%%%%%%%%%%%%%%%%%%%%%%%%%%%%%%%%%%%%%%%%%

\pagenumbering{Roman}
% --------------------------------------------------------------------------
%		Literaturverzeichnis / List of literature	
% -------------------------------------------------------------------------- 
\printbibliography[heading=bibintoc]

% --------------------------------------------------------------------------
%		Tabellenverzeichnis / List of tables
% -------------------------------------------------------------------------- 
\listoftables						% Tabellenverzeichnis / List of tables
\cleardoublepage

% --------------------------------------------------------------------------
%		Abbildungsverzeichnis / Register of illustrations
% -------------------------------------------------------------------------- 
\listoffigures						% Abbildungsverzeichnis / Register of illustrations
\cleardoublepage

% --------------------------------------------------------------------------
%		Anhang / Attachment
%
%		Der Anhnag enthält weitere Dokumente, welche nicht direkt zur Arbeit
%		gehören oder aus Platzgründen ausgelagert werden müssen. Die Kapitel
%		des Anhangs werden mit Großbuchstaben bezeichnet
%
%		The attachment contains further documents which are not related to work directly or
%		had to be outsourced due to a lack of space. The different chapters of the attachment 
%		are labeled with a capital letter.
% -------------------------------------------------------------------------- 
\appendix

% Inhalt des Anhangs (Beispiel) / List of contents of attachment (Example) 
\chapter{Illustrations}\label{cha:anhang_abbildungen}

\cleardoublepage
\chapter{Tables}\label{cha:anhang_tabellen}
\cleardoublepage

\chapter{Code}

\section{Build Script}\label{sec:apxblasm}

    \lstinputlisting[language=Bash]{05_appendix/blasm.sh}

\section{JavaScript Functions}\label{sec:apxmodule}

    \lstinputlisting[language=JavaScript]{05_appendix/module.js}

\section{RMW Adapter Function Headers}

    \subsection{Events}

        \lstinputlisting[language=C++]{05_appendix/rmwEvents.cpp}

    \subsection{Implementation Information}
        \lstinputlisting[language=C++]{05_appendix/rmwGetId.cpp}

        \lstinputlisting[language=C++]{05_appendix/rmwGetInfo.cpp}

        \lstinputlisting[language=C++]{05_appendix/rmwGetNames.cpp}

        \lstinputlisting[language=C++]{05_appendix/rmwGit.cpp}

    \subsection{Guard Conditions}

        \lstinputlisting[language=C++]{05_appendix/rmwGuard.cpp}

    \subsection{Initialization and Shutdown}

        % \lstinputlisting[language=C++]{05_appendix/rmwInit.cpp}

    % \lstinputlisting[language=C++]{05_appendix/rmwLog.cpp}

    % \lstinputlisting[language=C++]{05_appendix/rmwNode.cpp}

    % \lstinputlisting[language=C++]{05_appendix/rmwQos.cpp}

    % \lstinputlisting[language=C++]{05_appendix/rmwSerial.cpp}

    % \lstinputlisting[language=C++]{05_appendix/rmwSrvClt.cpp}

    % \lstinputlisting[language=C++]{05_appendix/rmwSrvSrv.cpp}

    % \lstinputlisting[language=C++]{05_appendix/rmwTopPub.cpp}

    % \lstinputlisting[language=C++]{05_appendix/rmwTopSub.cpp}

    % \lstinputlisting[language=C++]{05_appendix/rmwTopTake.cpp}

    % \lstinputlisting[language=C++]{05_appendix/rmwWait.cpp}

\backmatter
\end{document}