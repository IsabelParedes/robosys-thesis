\chapter{Middleware}

    A significant change from \ac{ROS} 1 to \ac{ROS} 2 is the shift from a custom transport layer consisting of \ac{TCPROS} to \ac{DDS}. \ac{DDS} is a publish-subscribe communication standard defined by \ac{OMG}. \ac{DDS} uses \ac{IDL} for defining and serializing messages~\cite{rosondds}. In contrast to \ac{ROS} 1, which requires a \ac{ROS} master in order for nodes to discover and communicate with each other, \ac{ROS} 2 discovery system is handled by \ac{DDS} and each of the \ac{DDS} vendors provides different options for customizing the communication layer.

    One notable advantage of moving away from a custom transport protocol is that the \ac{ROS} client libraries are now agnostic to the middleware interface; this means that the complexities of the \ac{DDS} implementation are not exposed to the end user~\cite{ros2middle}. As a consequence, multiple middleware interfaces can be implemented as long as they fulfill the following requirements: 
    \begin{itemize}
        \item publishing and subscribing
        \item message serialization
        \item discovery
    \end{itemize}
    
    The interaction between the \ac{ROS} user, the \ac{ROS} client libraries, and the middleware layers is shown in Figure~\ref{fig:middleware}.

    \begin{figure}[htbp]
        \centering
        \vspace{1em}
        \begin{tikzpicture}
            \node (user) [packBox] {ROS User};

            \node (rcl) [packBox, yshift=-2cm] {ROS Client Libraries \\ \small\textsf{rclcpp | rclpy}};

            \node (interface) [packBox, yshift=-4cm] {Middleware Interface \\ \small\textsf{rmw}};

            \node (implementation) [packBox, yshift=-6cm] {\textbf{Middleware Implementation} \\ \small\textsf{fastrtps | cyclonedds | connextdds | gurumdds | custom}};

            \begin{scope}[transform canvas={xshift=-0.5cm}]
                \draw [-to] (user) -- (rcl);
                \draw [-to] (rcl) -- (interface);
                \draw [-to] (interface) -- (implementation);
            \end{scope}

            \begin{scope}[transform canvas={xshift=0.5cm}]
                \draw [to-] (user) -- (rcl);
                \draw [to-] (rcl) -- (interface);
                \draw [to-] (interface) -- (implementation);
            \end{scope}

        \end{tikzpicture}
        \vspace{1em}
        \caption{Relations between the user, the \ac{ROS} client libraries and the middleware packages~\cite{ros2middle}.}
        \label{fig:middleware}
    \end{figure}


\section{Supported Implementations}

    Currently, \ac{ROS} 2 releases provide full support for three middleware implementations: eProsima Fast \ac{DDS}, Eclipse Cyclone \ac{DDS}, and \ac{RTI} Connext \ac{DDS}. The binaries also support Gurum \ac{DDS}, but the implementation requires a separate installation~\cite{docsdds}. 

    \subsection{eProsima Fast DDS}

    eProsima Fast \ac{DDS}, also known as Fast \ac{RTPS}, is the default middleware implementation for \ac{ROS} 2 packages. Some of the main advantages of Fast \ac{DDS} is that it is free, open source, and it is developed for most platforms including Linux, Windows, Mac OS, and QNX. A rich set of \ac{QoS} parameters is also available for tuning the communication protocols to any particular system. Fast \ac{DDS} follows a \ac{DCPS} model, which consists four elements: publishers, subscribers, topics, and domains~\cite{dcps}. This model introduces the concept of \textsf{Data Writers} and \textsf{Data Readers} which, as the names imply, have read and write permissions to the ``Global Data Space'' as specified by the \ac{DDS} standard~\cite{introdds}. Figure~\ref{fig:ddsdomain} displays an example of the Fast \ac{DDS} architecture and demonstrates how the different elements interact with each other.

    \begin{figure}[htbp]
        \centering
        \vspace{1em}
        \begin{tikzpicture}
            \node (domain) [
                box, 
                minimum width=14cm,
                text depth=6cm,
                fill=igmrLightBlue!10!bgColor,
            ] {DDS Domain};
            
            \node (p1) [
                box,
                xshift=-4cm,
                minimum width = 4cm,
                text depth=4.5cm,
                fill=igmrLightBlue!40!bgColor,
            ] {Domain Participant};

            \node (pub1) [
                box,
                xshift=-4cm,
                yshift=0.5cm,
                minimum width=3.5cm,
                text depth=1cm
            ] {Publisher};

            \node (dw1) [
                box,
                xshift=-4cm,
                yshift=0.2cm,
                minimum width=3cm,
                fill=bgColor,
            ] {Data Writer};

            \node (sub1) [
                box,
                xshift=-4cm,
                yshift=-1.5cm,
                minimum width=3.5cm,
                text depth=1cm
            ] {Subscriber};

            \node (dr1) [
                box,
                xshift=-4cm,
                yshift=-1.8cm,
                minimum width=3cm,
                fill=bgColor,
            ] {Data Reader};

            \node (p2) [
                box,
                xshift=4cm,
                minimum width=4cm,
                text depth=4.5cm,
                fill=igmrLightBlue!40!bgColor,
            ] {Domain Participant};

            \node (sub2) [
                box,
                xshift=4cm,
                yshift=0.5cm,
                minimum width=3.5cm,
                text depth=1cm
            ] {Subscriber};

            \node (dr2) [
                box,
                xshift=4cm,
                yshift=0.2cm,
                minimum width=3cm,
                fill=bgColor,
            ] {Data Reader};

            \node (t1) [
                box,
                minimum width=2cm,
                minimum height=1cm,
                fill=igmrLightBlue!40!bgColor,
            ] {Topic};

            \node (dot1) [
                rectangle,
                xshift=-1.5cm,
            ] {};

            \draw [thick] (t1) -- (dot1.center);
            \draw [arrow] (dw1.east) -- (dw1.east-|t1.west);
            \draw [arrow] (dot1.center) |- (dr1);
            \draw [arrow] (t1.east|-dr2.west) -- (dr2.west);

        \end{tikzpicture}
        \vspace{1em}
        \caption{Instance of a typical Fast \ac{DDS} domain model.}
        \label{fig:ddsdomain}
    \end{figure}

    \subsection{Eclipse Cyclone DDS}

        Similar to Fast \ac{DDS}, Cyclone \ac{DDS} is free and open source and supports the three major platforms, Linux, Windows, and Mac OS. Cyclone \ac{DDS} offers a ``data-centric'' architecture with space- and time-decoupling with a zero configuration discovery system~\cite{eclipse}. Additionally, this implementation includes Python bindings to simplify the definition of data types.

    \subsection{RTI Connext DDS}

        \ac{RTI}

    \subsection{GurumNetworks Gurum DDS}

    


\section{Custom Middleware}

    Why it needs to be replaced

    \subsection{Email}

    \subsection{Zenoh}

    Minimal implementation (minimal set of functions)

    
\section{Substituting ROS 2 Middleware}

    At run time

    At build time

\section{Custom Middleware Design}

    Design of middleware packages (tree diagram or something)

    \subsection{\textsf{wasm\_cpp}}


    \begin{figure}[htbp]
        \centering
        \begin{tikzpicture}%[show background grid]
            \begin{abstractclass}[text width=5cm]{Participant}{0,0}
                \attribute{- name : String}
                \attribute{- role : String}
                \attribute{- gid  : String}

                \operation{- is\_valid\_name()}
                \operation{- is\_valid\_role()}
                \operation{- registration()}
                \operation{- deregistration()}
            \end{abstractclass}

            \begin{class}[text width=5cm]{Publisher}{-5,-6}
                \inherit{Participant}
                \attribute{- name = topic\_name}
                \attribute{- role = publisher}
                \operation{+ publish(message : String)}
            \end{class}

            \begin{class}[text width=5cm]{Subscriber}{5,-6}
                \inherit{Participant}
                \attribute{- name = topic\_name}
                \attribute{- role = subscriber}
                \operation{+ get\_message() : String}
            \end{class}

            \begin{class}[text width=6.5cm]{ServiceServer}{-4,-11}
                \inherit{Participant}
                \attribute{- name = service\_name}
                \attribute{- role = service\_server}
                \operation{+ take\_request() : String}
                \operation{+ send\_response(response : String)}
            \end{class}

            \composition{ServiceServer}{}{}{Publisher}
            \composition{ServiceServer}{}{}{Subscriber}

            \begin{class}[text width=6.5cm]{ServiceClient}{4,-11}
                \inherit{Participant}
                \attribute{- name = service\_name}
                \attribute{- role = service\_client}
                \operation{+ send\_request(request : String)}
                \operation{+ take\_response() : String}
                \operation{+ is\_service\_available() : Bool}
            \end{class}

            \composition{ServiceClient}{}{}{Publisher}
            \composition{ServiceClient}{}{}{Subscriber}

        \end{tikzpicture}
        \caption{A class diagram}
    \end{figure}
    
