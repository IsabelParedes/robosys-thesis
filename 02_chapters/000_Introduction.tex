\chapter{Introduction}\label{cha:introduction}

In recent years, \ac{ROS} has become the de-facto standard for robotics. Thanks to the wide variety of libraries and the active community, \ac{ROS} has been integrated into a wide range of applications ranging from education to industrial processes. With an increasing number of users come concerns regarding who can access \ac{ROS} distributions.

Recently, \ac{ROS} has become more accessible for the average robotics enthusiast; and with the release of \ac{ROS} 2, users can now have a native \ac{ROS} installation on the most common platforms, Ubuntu, Windows, and macOS. Additionally, these distributions can also be installed in a \textsf{conda} environment thanks to the efforts by the RoboStack team. However, there can be many variations in package versions which make replicating workspaces with users of other platforms difficult. 

An example of a situation in which replicating a workspace would be useful is in a robotics course. Teaching \ac{ROS} poses its own set of challenges, one of which is to provide the students with all of the tools they require to succeed in the course. In a \ac{ROS} course, all the students would need to have access to a \ac{ROS} installation. There are at least two methods to accomplish this, one is to setup and maintain a computer lab that the students can regularly access, and the second is to provide instructions so that the students can replicate the instructor's workspace on their own personal computers. However, the issue with providing instructions is that students will have different computers and some of the \ac{ROS} packages may not work the same for everyone. One workaround to this problem has been to use Docker containers, but installing Docker and launching the containers can still be an intimidating task for aspiring roboticists. 


An alternative solution to Docker containers is proposed in this project, and it revolves around \ac{WASM}. Instead of creating Docker images for sharing or maintaining computer labs, it is proposed that a \ac{ROS} 2 workspace can be packaged for any web browser and distributed with a link. By cross-compiling \ac{ROS} packages to WebAssembly, this project explores the potential of running \ac{ROS} nodes on the browser and experiments with the communications between them. Unlike other \ac{ROS} tools which rely on a connection to a \ac{ROS} system or a server, this proposal will create an isolated environment running on the browser. 

The solution proposed is divided into three main tasks: creation of a middleware implementation which works well with WebAssembly, cross-compilation of packages to WebAssembly, and the deployment of a website containing the \ac{ROS}\footnote{In the following chapters, \ac{ROS} refers to \ac{ROS} 2, particularly \ac{ROS} 2 Humble, unless otherwise stated.} packages to be shared with the reader. Chapters~\ref{cha:literagure} and~\ref{cha:concept} describe related projects and formulate the concept for porting \ac{ROS} to WebAssembly. Chapter~\ref{cha:methodology} focuses on the procedures and tools used to complete this project. Chapter~\ref{cha:middleware} and Chapter~\ref{cha:rmw} describe the design of the middleware implementation. And lastly, the assessment of all of the elements developed is done in Chapter~\ref{cha:assessment}.