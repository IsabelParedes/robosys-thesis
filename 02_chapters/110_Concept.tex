\chapter{Concept Realization}\label{cha:concept}


\section{Concept}\label{sec:concept}

% Goals:
% - open source

% Ideal scenario: 
% - click on a link and run ROS
% - connect to a robot via bluetooth
% - share simulations and algorithms

    \subsection{Target Scenario}

        The goals for this project can be defined by an ideal scenario in which 
        a high level of usability is reached by an intermediate user. First, an
        intermediate user is described as an individual who is familiar with the 
        ROS ecosystem but does not have the need to maintain or test ROS packages
        across different platforms. In the target scenario, this intermediate
        user will be capable of performing the following tasks:

        \begin{itemize}
            \item install pre-compiled ROS 2 packages in the browser
            \item launch nodes including publishers, subscribers, servers, and clients
            \item interact with the environment to obtain information about 
                  running nodes, this would include echoing topics, listing 
                  parameters, reviewing log files, etc.
            \item visualization of URDFs, transforms, point clouds, markers, etc.
            \item playing and recording bag files % ???
            \item connecting with robots via bluetooth
        \end{itemize}




\section{Implementation Layers}

    \subsection{User Levels}

        \begin{table}[htbp]
            \color{textColor}
            \centering	
            \caption{Target users categorized by expertise level.}
                \begin{tabular}{rll}
                    \toprule
                    & \textbf{User}   & \textbf{Description} \\
                    \midrule
                    $0$ & Beginner    & Complete beginners who have never used ROS or programmed \\
                    & & in any language. \\[0.5em]

                    $1$ & Student     & University students with basic programming experience. \\[0.5em]

                    $2$ & ROS User    & Students and researchers who actively use ROS for projects. \\[0.5em]

                    $3$ &  Roboticist & Robotics software developers including contributors to the \\
                    & & ROS ecosystem. \\
                \bottomrule
            \end{tabular}\label{tab:userlevels}
        \end{table}

    \subsection{User Levels of Interaction}

        \begin{table}[htbp]
            \color{textColor}
            \centering	
            \caption{User interface segmented based on the level of interaction.}
                \begin{tabular}{rll}
                    \toprule
                    & \textbf{Interface} & \textbf{Description} \\
                    \midrule
                    $0$ & Non-interactive & Nodes run automatically as soon as the site is launched. \\ [0.5em]

                    $1$ & Minimal         & User can start/stop 1$-$2 nodes by pressing a button. \\[0.5em]

                    $2$ & Basic           & User can select which nodes to run and can analyze the \\
                    & & environment by requesting or viewing information. \\[0.5em]

                    $3$ & Intermediate    & The graphical interface allows the user to accomplish \\
                    & & primary tasks, such as displaying a robot. \\[0.5em]

                    $4$ & Advanced        & A complete GUI where the user has full control of the \\
                    & & environment, can start/stop nodes, modify params, \\
                    & & interact with robots, etc. \\[0.5em]

                    $5$ & Complete        & All ROS 2 features are available and packages can be \\
                    & & built on the browser \\

                \bottomrule
            \end{tabular}\label{tab:uilevels}
        \end{table}

    \subsection{Technical Levels}

        \begin{table}[htbp]
            \color{textColor}
            \centering	
            \caption{Implementation categories with increasing technical difficulty.}
                \begin{tabular}{cl}
                    \toprule
                    \textbf{Level} & \textbf{Description} \\
                    \midrule
                    $0$ & A publisher is displayed. \\ [0.3em]
                    $1$ & A publisher and subscriber can communicate with each other and \\
                        & offer minimal interaction to start and stop each node. \\[0.3em]
                    $2$ & Multiple nodes and distinct topics with limited interaction. \\[0.3em]
                    $3$ & Graphical display and interaction with a ROS client library. \\[0.3em]
                    $4$ & Manipulation of a physical robot wirelessly. \\[0.3em]
                    $5$ & Visualization of a robot with Zethus. \\[0.3em]
                    $6$ & Simulation of a robotics scenario with Gazebo. \\[0.3em]
                    $7$ & Development workspace for creating and debugging ROS packages. \\
                \bottomrule
            \end{tabular}\label{tab:techlevels}
        \end{table}

\section{Scope of Implementation}

    \subsection{Middleware Replacement}
        % Why it needs to be replaced
        % why sockets don't work

        % 2 Parts:
        % Interface with ROS
        % JS ROS master to handle communications

    \subsection{User Interface}

        % Simple website
        % JupyterLite

    \subsection{Automatic Builds}
        
        % GitHub Actions for automatically building packages


    \subsection{Out of Scope}
        
        % Gazebo
        % Simulation
        % In-browser compilation

