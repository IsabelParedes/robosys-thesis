\chapter{Concept Realization}\label{cha:concept}

    This section provides the major milestones from the project beginning with a
    brief description of the overall concept solution to the challenges presented
    in the Introduction, followed by the layers of implementation accomplished 
    during the development phase.


\section{Target Scenario}\label{sec:target}

% Goals:
% - open source

% Ideal scenario: 
% - click on a link and run ROS
% - connect to a robot via bluetooth
% - share simulations and algorithms

    To introduce the concept, a ``target scenario'' is first considered. In this
    scenario, an intermediate \ac{ROS} user should be able to reach a high level 
    of usability with the tools developed in this project. First, an
    intermediate user is described as an individual who is familiar with the 
    \ac{ROS} ecosystem but does not have the need to maintain or test \ac{ROS} packages
    across different platforms. In the target scenario, this intermediate
    user will be capable of performing the following tasks:

    \begin{itemize}
        \item install pre-compiled ROS 2 packages in the browser
        \item launch nodes including publishers, subscribers, servers, and clients
        \item interact with the environment to obtain information about 
                running nodes, this would include echoing topics, listing 
                parameters, reviewing log files, etc.
        \item visualization of \ac{URDF} files, transforms, point clouds, markers, etc.
        \item playing and recording bag files % ???
        \item connecting with robots via bluetooth
    \end{itemize}

    Outside of this scenario, another goal for this project includes making the 
    developed tools available to the general public by distributing them as
    open-source software. This will allow other roboticists to compile their own
    packages and share them on the web.


\section{Implementation Layers}

    \textcolor{red}{TODO: add some introduction to these layers} 




    \begin{tcolorbox}[title=Note]
        \begin{minipage}[t]{0.87\linewidth}
            \vspace*{0pt}
            If the reader would like to follow along with the demonstrations
            provided in the following pages, it is recommended to visit 
            \href{https://ros2wasm.dev/}{\texttt{ros2wasm.dev}}.
            Throughout the text, links will be provided to redirect the reader 
            to specific examples.
        \end{minipage}\hfill%
        \begin{minipage}[t]{0.1\linewidth}
            \vspace*{0pt}
            \includegraphics[height=\linewidth,width=\linewidth]{qr_ros2wasm.png}
        \end{minipage}
    \end{tcolorbox}


    \subsection{User Levels}

        For the purpose of establishing target users for the developed tools,
        potential users were categorized based on expertise level with \ac{ROS} 
        and programming in general, as observed in Table~\ref{tab:userlevels}.

        Commencing with Level 0, the \textit{Beginner} category is reserved for
        students in secondary education who have had little to no experience with
        programming, and therefore are not familiar with \ac{ROS}. The tools
        developed in this project would serve as an initial introduction to 
        robotics for this category of users.

        Level 1 consists of university students who have completed elementary 
        programming courses but have not yet been introduced to \ac{ROS}. For this
        type of user, this project will provide essential tutorials to become
        acquainted with the inner workings of \ac{ROS}.

        With a slightly higher level of expertise, Level 2 comprises students or
        other enthusiasts who are already familiar with \ac{ROS} and have collaborated
        in projects which use \ac{ROS} as the main system to handle communications
        of multiple robotics elements. This ROS user is equivalent to the intermediate
        user described in the target scenario (Section~\ref{sec:target}).

        Lastly, the highest level of experience is dedicated to roboticists who
        actively use and contribute to the development of \ac{ROS}. For this 
        category of users, the intention of this project will be to involve more
        contributors in order to more promptly meet the needs of most \ac{ROS} users.



        \begin{table}[htbp]
            \color{textColor}
            \centering	
            \caption{Target users categorized by expertise level.}
                \begin{tabular}{rll}
                    \toprule
                    & \textbf{User}   & \textbf{Description} \\
                    \midrule
                    $0$ & Beginner    & Complete beginners who have never used ROS or programmed \\
                    & & in any language. \\[0.5em]

                    $1$ & Student     & University students with basic programming experience. \\[0.5em]

                    $2$ & ROS User    & Students and researchers who actively use ROS for projects. \\[0.5em]

                    $3$ &  Roboticist & Robotics software developers including contributors to the \\
                    & & ROS ecosystem. \\
                \bottomrule
            \end{tabular}\label{tab:userlevels}
        \end{table}

    \subsection{User Levels of Interaction}

        \ac{GUI}

        \begin{table}[htbp]
            \color{textColor}
            \centering	
            \caption{User interface segmented based on the level of interaction.}
                \begin{tabular}{rll}
                    \toprule
                    & \textbf{Interface} & \textbf{Description} \\
                    \midrule
                    $0$ & Non-interactive & Nodes run automatically as soon as the site is launched. \\ [0.5em]

                    $1$ & Minimal         & User can start/stop 1$-$2 nodes by pressing a button. \\[0.5em]

                    $2$ & Basic           & User can select which nodes to run and can analyze the \\
                    & & environment by requesting or viewing information. \\[0.5em]

                    $3$ & Intermediate    & The graphical interface allows the user to accomplish \\
                    & & primary tasks, such as displaying a robot. \\[0.5em]

                    $4$ & Advanced        & A complete GUI where the user has full control of the \\
                    & & environment, can start/stop nodes, modify params, \\
                    & & interact with robots, etc. \\[0.5em]

                    $5$ & Complete        & All ROS 2 features are available and packages can be \\
                    & & built on the browser \\

                \bottomrule
            \end{tabular}\label{tab:uilevels}
        \end{table}

    \subsection{Technical Levels}

        \begin{table}[htbp]
            \color{textColor}
            \centering	
            \caption{Implementation categories with increasing technical difficulty.}
                \begin{tabular}{cl}
                    \toprule
                    \textbf{Level} & \textbf{Description} \\
                    \midrule
                    $0$ & A publisher is displayed. \\ [0.3em]
                    $1$ & A publisher and subscriber can communicate with each other and \\
                        & offer minimal interaction to start and stop each node. \\[0.3em]
                    $2$ & Multiple nodes and distinct topics with limited interaction. \\[0.3em]
                    $3$ & Graphical display and interaction with a ROS client library. \\[0.3em]
                    $4$ & Manipulation of a physical robot wirelessly. \\[0.3em]
                    $5$ & Visualization of a robot with Zethus. \\[0.3em]
                    $6$ & Simulation of a robotics scenario with Gazebo. \\[0.3em]
                    $7$ & Development workspace for creating and debugging ROS packages. \\
                \bottomrule
            \end{tabular}\label{tab:techlevels}
        \end{table}

% \section{Scope of Implementation}

    % \subsection{Middleware Replacement}
        % Why it needs to be replaced
        % why sockets don't work

        % 2 Parts:
        % Interface with ROS
        % JS ROS master to handle communications

    % \subsection{User Interface}

        % Simple website
        % JupyterLite

    % \subsection{Automatic Builds}
        
        % GitHub Actions for automatically building packages


    % \subsection{Out of Scope}
        
        % Gazebo
        % Simulation
        % In-browser compilation

