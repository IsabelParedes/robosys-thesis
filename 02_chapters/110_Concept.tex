\chapter{Concept Realization}\label{cha:concept}


\section{Concept}\label{sec:concept}

Ideal scenario: 
- click on a link and run ROS
- connect to a robot via bluetooth
- share simulations and algorithms

\section{Implementation Layers}

    \subsection{User Levels}

        \begin{table}[htbp]
            \centering	
            \caption{TODO:}
                \begin{tabular}{ll}
                    \toprule
                    \textbf{User} & \textbf{Description} \\
                    \midrule
                    Beginners   & Complete beginners who have never used ROS or programmed \\
                                & in any language. \\[0.5em]
                    Students    & University students with minimal programming experience. \\[0.5em]
                    ROS Users   & Students and researchers who actively use ROS for projects. \\[0.5em]
                    Roboticists & Robotics software developers including contributors to the \\
                                & ROS ecosystem. \\
                \bottomrule
            \end{tabular}\label{tab:userlevels}
        \end{table}

    \subsection{User Levels of Interaction}

        \begin{table}[htbp]
            \centering	
            \caption{TODO:}
                \begin{tabular}{ll}
                    \toprule
                    \textbf{UI Level} & \textbf{Description} \\
                    \midrule
                    Non-interactive & Nodes run automatically as soon as the site is launched. \\ [0.5em]

                    Minimal         & User can start/stop 1$-$2 nodes by pressing a button. \\[0.5em]

                    Basic           & User can select which nodes to run and can analyze the \\
                                    & environment by requesting or viewing information. \\[0.5em]

                    Intermediate    & The graphical interface allows the user to accomplish \\
                                    & primary tasks, such as displaying a robot. \\[0.5em]

                    Advanced        & A complete GUI where the user has full control of the \\
                                    & environment, can start/stop nodes, modify params, \\
                                    & interact with robots, etc. \\[0.5em]
                                    
                    Complete        & All ROS2 features are available and packages can be \\
                                    & built on the browser \\
                \bottomrule
            \end{tabular}\label{tab:uilevels}
        \end{table}

    \subsection{Technical Levels}

        \begin{table}[htbp]
            \centering	
            \caption{TODO:}
                \begin{tabular}{cl}
                    \toprule
                    \textbf{Level} & \textbf{Description} \\
                    \midrule
                    L0 & A publisher and subscriber example on the console. \\ [0.3em]
                    L1 & Multiple nodes and topics with limited interaction. \\[0.3em]
                    L2 & Graphical display with user interaction. \\[0.3em]
                    L3 & Simulation of a robot (urdf). \\[0.3em]
                    L4 & Manipulation of physical robot. \\[0.3em]
                    L5 & Simulation of a robotics scenario. \\
                \bottomrule
            \end{tabular}\label{tab:techlevels}
        \end{table}

\section{Scope}

    Middleware replacement (why sockets don't work)

    JavaScript ``ROS master''

